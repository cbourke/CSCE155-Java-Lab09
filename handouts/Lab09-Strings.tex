\documentclass[12pt]{scrartcl}



\setlength{\parindent}{0pt}
\setlength{\parskip}{.25cm}

\usepackage{graphicx}

\usepackage{xcolor}

\definecolor{darkred}{rgb}{0.5,0,0}
\definecolor{darkgreen}{rgb}{0,0.5,0}
\usepackage{hyperref}
\hypersetup{
  letterpaper,
  colorlinks,
  linkcolor=red,
  citecolor=darkgreen,
  menucolor=darkred,
  urlcolor=blue,
  pdfpagemode=none,
  pdftitle={CS1 - Lab 9.0 - Java},
  pdfauthor={Christopher M. Bourke},
  pdfkeywords={}
}

\definecolor{MyDarkBlue}{rgb}{0,0.08,0.45}
\definecolor{MyDarkRed}{rgb}{0.45,0.08,0}
\definecolor{MyDarkGreen}{rgb}{0.08,0.45,0.08}

\definecolor{mintedBackground}{rgb}{0.95,0.95,0.95}
\definecolor{mintedInlineBackground}{rgb}{.90,.90,1}

%\usepackage{newfloat}
\usepackage[newfloat=true]{minted}
\setminted{mathescape,
               linenos,
               autogobble,
               frame=none,
               framesep=2mm,
               framerule=0.4pt,
               %label=foo,
               xleftmargin=2em,
               xrightmargin=0em,
               startinline=true,  %PHP only, allow it to omit the PHP Tags *** with this option, variables using dollar sign in comments are treated as latex math
               numbersep=10pt, %gap between line numbers and start of line
               style=default, %syntax highlighting style, default is "default"
               			    %gallery: http://help.farbox.com/pygments.html
			    	    %list available: pygmentize -L styles
               bgcolor=mintedBackground} %prevents breaking across pages
               
\setmintedinline{bgcolor={mintedBackground}}
\setminted[text]{bgcolor={mintedBackground},linenos=false,autogobble,xleftmargin=1em}
%\setminted[php]{bgcolor=mintedBackgroundPHP} %startinline=True}
\SetupFloatingEnvironment{listing}{name=Code Sample}
\SetupFloatingEnvironment{listing}{listname=List of Code Samples}


\title{CSCE 155 - Java}
\subtitle{Lab 9.0 - Strings}
\author{~}
\date{~}

\begin{document}

\maketitle

\section*{Prior to Lab}

Before attending this lab:
\begin{enumerate}
  \item Read and familiarize yourself with this handout.
  \item Review Oracle's documentation about Strings in Java: 
  	\url{http://docs.oracle.com/javase/tutorial/java/data/strings.html}
\end{enumerate}

\section*{Peer Programming Pair-Up}

To encourage collaboration and a team environment, labs will be
structured in a \emph{pair programming} setup.  At the start of
each lab, you will be randomly paired up with another student 
(conflicts such as absences will be dealt with by the lab instructor).
One of you will be designated the \emph{driver} and the other
the \emph{navigator}.  

The navigator will be responsible for reading the instructions and
telling the driver what to do next.  The driver will be in charge of the
keyboard and workstation.  Both driver and navigator are responsible
for suggesting fixes and solutions together.  Neither the navigator
nor the driver is ``in charge.''  Beyond your immediate pairing, you
are encouraged to help and interact and with other pairs in the lab.

Each week you should alternate: if you were a driver last week, 
be a navigator next, etc.  Resolve any issues (you were both drivers
last week) within your pair.  Ask the lab instructor to resolve issues
only when you cannot come to a consensus.  

Because of the peer programming setup of labs, it is absolutely 
essential that you complete any pre-lab activities and familiarize
yourself with the handouts prior to coming to lab.  Failure to do
so will negatively impact your ability to collaborate and work with 
others which may mean that you will not be able to complete the
lab.  

\section{Lab Objectives \& Topics}
At the end of this lab you should be familiar with the following
\begin{itemize}
  \item Understand the way Java implements and handles Strings
  \item Use methods of String and understand the difference between 
  	classes \mintinline{java}{String} and \mintinline{java}{StringBuilder}
\end{itemize}

\section{Background}

A string in Java is a collection of characters regarded and handled as 
a single unit.  The \mintinline{java}{String} class is used to represent 
strings, so they are objects not primitive types.  Like other objects, 
strings have their own set of constructors and methods.  When used
with methods, strings are passed by reference not by value.  However, 
unlike most other objects Strings can be instantiated without using 
the \mintinline{java}{new} operator.  For example, the following lines of 
code are all valid string declarations:

\begin{minted}{java}
String myString = "This is the string lab for CSE 155";
String myOtherString = new String( "This is the string lab for CSE 155" );
String anotherString = null;
String emptyString = "";
\end{minted}

The reference variable \mintinline{java}{myString} refers to a string literal 
and \mintinline{java}{anotherString} is a reference that points to \mintinline{java}{null}.
However, it can still be used to eventually reference a valid \mintinline{java}{String}
object.  The reference variable \mintinline{java}{emptyString} does 
\emph{not} point to \mintinline{java}{null}, it is the empty string, and simply 
points to a \mintinline{java}{String} object that has no characters.  In Java 
strings are immutable.  That is, once created their contents cannot be changed.  
\mintinline{java}{StringBuilder} is a \mintinline{java}{String}-like class that 
provides methods to change character contents.

See the Java API for a full description of the \mintinline{java}{String} 
and \mintinline{java}{StringBuilder} classes: 
\begin{itemize}
  \item \url{http://docs.oracle.com/javase/6/docs/api/java/lang/String.html}.
  \item \url{http://docs.oracle.com/javase/1.5.0/docs/api/java/lang/StringBuilder.html}
\end{itemize}

\section{Activities}

This lab will familiarize you with some of these concepts.  In particular, you 
will complete a program that implements a common children?s game, horse 
(also known as hangman).  In this game, an English word is chosen at random 
and its characters hidden.  The player takes turns by guessing a letter; each 
instance (if any) of the guessed letter is revealed.  If the user is able to guess 
the word before a certain number of guessed letters then they win.  If they run 
out of guesses then they lose.

Most of the game mechanics have been implemented for you.  However, you 
will need to complete the game by implementing several methods used by the 
game to manipulate and compare Strings.  Clone the project for this lab from
GitHub using the following URL: \url{https://github.com/cbourke/CSCE155-Java-Lab09}.

\subsection{Implementing String Manipulation Methods}

\begin{enumerate}
  \item Examine the content of \mintinline{text}{ObtainInput.java} and 
  	\mintinline{text}{PlayHorse.java} but do not change them.
  \item Open \mintinline{text}{HorseGame.java} in Eclipse.  There are several 
	methods already fully implemented in this file.  Your task for this lab is to 
	implement the following four methods:
	\begin{itemize}
	  \item \mintinline{java}{initializeBlankString()} - This method takes one variable 
	  as input: a \mintinline{java}{String} variable containing the secret word.  It 
	  should return nothing.   The method should alter the \mintinline{java}{StringBuilder} 
	  instance so that it is of equal length to the secret word and set all of its values to 
	  an underscore.  The \mintinline{java}{StringBuilder} instance will slowly reveal 
	  correctly guessed letters
	  \item \mintinline{java}{printWithSpaces()} - This method will print the contents 
	  of the \mintinline{java}{StringBuilder} instance with spaces between each character.   
	  The method requires no input and should return nothing.
	  \item \mintinline{java}{revealGuessedLetter()} - This method will take a 
	  \mintinline{java}{String} (the secret word) and a character (the user's guess) as input.   
	  The method should alter the \mintinline{java}{StringBuilder} instance in the 
	  following way: for every position in the secret word that contains the character 
	  passed in as the second argument, change the same position in the 
	  \mintinline{java}{StringBuilder} instance to that character.  For example, if the 
	  secret word is: \mintinline{java}{"dinosaur"} and the \mintinline{java}{StringBuilder}
	  variable is: \mintinline{java}{"________"} and the character passed is: \mintinline{java}{a}
	  the method should alter the \mintinline{java}{StringBuilder} so that it becomes 
	  \mintinline{java}{"_____a__"}.  The method should return \mintinline{java}{true} 
	  if any letters were changed and \mintinline{java}{false} otherwise.
	  \item \mintinline{java}{checkGuess()} - This method checks if the 
	  \mintinline{java}{StringBuilder} instance is the same as the secret word string.  
	  The method should return \mintinline{java}{true} if they are, and 
	  \mintinline{java}{false} otherwise.
	\end{itemize}
  \item Complete the implementation of the methods, and run the program in Eclipse.
  \item Answer the questions on your worksheet.
\end{enumerate}

\section{Handin/Grader Instructions}

\begin{enumerate}
  \item Hand in your completed files:
    \begin{itemize}
    \item \mintinline{text}{HorseGame.java}
    \item \mintinline{text}{ObtainInput.java}
    \item \mintinline{text}{PlayHorse.java}
    \item \mintinline{text}{worksheet.md}
  \end{itemize}
  through the webhandin (\url{https://cse-apps.unl.edu/handin}) 
  using your cse login and password.  
  \item Even if you worked with a partner, you \emph{both} should
  turn in all files.
  \item Verify your program by grading yourself through the
  webgrader (\url{https://cse.unl.edu/~cse155h/grade/}) using the
  same credentials.
  \item Recall that both expected output and your program's output
  will be displayed.  The formatting may differ slightly which is fine.
  As long as your program successfully compiles, runs and outputs 
  the \emph{same values}, it is considered correct.
\end{enumerate}


\section{Advanced Activity (Optional)}

If strings are immutable, and their contents cannot be changed, why does the 
following code execute as expected?  

\begin{minted}{java}
String myString = "This week we?re studying Strings!"; 
System.out.println("myString contains: " + myString);
myString = "Next week we?re studying file I/O!";
System.out.println("myString now contains: " + myString);
\end{minted}

Did the contents of \mintinline{java}{myString} change?
Suppose we add another line of code as follows:

\mintinline{java}{System.out.println( "myString now contains: "+ myString.toUpperCase() );}

What did the above line of code do?  Did it also change the contents 
of \mintinline{java}{myString}?

\end{document}
